\documentclass{article}

% Packages
\usepackage{amsmath} % For mathematical equations
\usepackage{graphicx} % For including figures
\usepackage{cite} % For citations
\usepackage{lipsum} % For generating dummy text

% Title
\title{In house high Q Electro-Optic Modulator for Molecular Imaging }
\author{A. C. Baldock}
\date{\today}

\begin{document}

\maketitle

\begin{abstract}
    % Abstract of your paper
\end{abstract}

\section{Introduction}
This paper outlines the theory and the method to construct a resonant
electro-optic modular (EOM) for molecular imaging. 

\section{Theory}
An electro-optic phase modulator is a device which converts an time-varying
electric field (typically in the RF frequency) into a modulate of the phase of an
optical frequency. This is achieved by sending linearly polarized light along
the an axis of a nonlinear crystal while modulating the crystal's index of
refraction \(n_e\).

Appling a voltage $V$ across the crystal will induce a phase shift, \(\Delta\phi\), of~\cite{Mok}
\begin{equation}
    \Delta\phi = \frac{\pi n_e^3r_{33}}{\lambda}\frac{V l}{d}
    \label{eq:phase_shift}
\end{equation}.

$r_{33}$ is the electro-optic coefficient which is defined by the choice of crystal, \(\lambda\) is the wavelength of the
light, \(l\) is the length of the crystal, and \(d\) is the distance between the
electrodes.

EOMs are characterised by the half wave voltage which results in a phase shift of \(\Delta\phi=\pi\). This is given by
\begin{equation}
    V_{\pi} = \frac{\lambda d}{n_e^3r_{33}l}.
    \label{eq:half_wave_voltage}
\end{equation}

Treating the crystal as a distributed capacitor we can design a LCR tank circuit
to resonate at the desired frequency. The resonant frequency is given by,
\begin{equation}
    f_0 = \frac{1}{2\pi\sqrt{LC}},
    \label{eq:resonant_frequency}
\end{equation}
where \(C\) is the capacitance of the crystal and \(L\) is the inductance of the
tank circuit which we select to define the operational frequency of the EOM.
Driving the EOM at the resonant frequency from an RF source will result in the 
maximim power transfer into the crystal and hence the maximum phase modulation.

Applying a time varying voltage across the crystal induces a time dependence
onto the refractive index of the crystal. This results in a phase modulation
which produces sidebands around the carrier frequency $\Omega$ which the separation
dictated the RF driving frequency, $\omega$. The amplitude of the sidebands are given by,
\begin{equation}
    \mathbf{E}(t) = \mathbf{E}_0\sum_{n=-\infty}^{\infty}J_n(M)e^{it(\Omega+n\omega)},
\end{equation}
where $\mathbf{E}_0$ is the amplitude of the carrier, $J_n$ is the Bessel
function of the $n$ kind, and $M$ is the modulation index given by,
\begin{equation}\label{eq:modulation_index}
    M = \frac{\pi V}{V_{\pi}},
\end{equation}
wehre $V$ is the amplitude of the driving voltage. For typical crystal specifications it 
is easy to produce the first order sidebands, even with a modulation index of $M\ll 1$. 
To produce higher order sidebands it is evident from equation \ref{eq:modulation_index} that
a higher driving voltage is required. 

It is typical to use a RF amplifier to increase to driving voltage to get the
desired modulation index. An important note is in a resonant LCR tank circuit
the Q-factor increases the voltage across the crystal by a factor of Q, so it is
imperative to design a EOM with a high Quality factor.

In order to minimise the power reflection into the EOM crystal it is important
to match the impedance of the driving source, typically 50$~\Omega$, to the
characteristic impedance of the LCR circuit. This duty can be preformed by
a transformer whos turns ratio is given by,
\begin{equation}
    \frac{N_1}{N_2} = \sqrt{\frac{Z_1}{Z_2}},
\end{equation}
where $N_1$ and $N_2$ are the number of turns in the primary and secondary and
$Z_1$ and $Z_2$ are the characteristic impedances of the driving source and the
LCR circuit respectively.

\section{Construction}

Our EOM uses a 5\% MgO doped LiNbO$_3$ crystal with a length of 40mm and a width
and height of 3mm. The crystal has a refractive index of 2.2 and a electro-optic
coefficient of $r_{33}=30.9$ pm/V. This was mounted on a electrically isolated
acrylic block, and secured with two copper electrodes. The electrodes were
sufficiently large enough to act as a heat sink for the crystal in order to
minimise thermal induced lensing effects.


\begin{thebibliography}{9}
    \bibitem{Mok}
    C Mok, M Weel, E Rotberg, and A Kumarakrishnan. Design and construction of
    an efficient electro-optic modulator for laser spectroscopy (2016). Canadian
    Journal of Physics. 84(9): 775-786.
\end{thebibliography}    



\end{document}
